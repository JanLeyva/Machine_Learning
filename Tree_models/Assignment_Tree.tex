% Options for packages loaded elsewhere
\PassOptionsToPackage{unicode}{hyperref}
\PassOptionsToPackage{hyphens}{url}
%
\documentclass[
]{article}
\usepackage{lmodern}
\usepackage{amssymb,amsmath}
\usepackage{ifxetex,ifluatex}
\ifnum 0\ifxetex 1\fi\ifluatex 1\fi=0 % if pdftex
  \usepackage[T1]{fontenc}
  \usepackage[utf8]{inputenc}
  \usepackage{textcomp} % provide euro and other symbols
\else % if luatex or xetex
  \usepackage{unicode-math}
  \defaultfontfeatures{Scale=MatchLowercase}
  \defaultfontfeatures[\rmfamily]{Ligatures=TeX,Scale=1}
\fi
% Use upquote if available, for straight quotes in verbatim environments
\IfFileExists{upquote.sty}{\usepackage{upquote}}{}
\IfFileExists{microtype.sty}{% use microtype if available
  \usepackage[]{microtype}
  \UseMicrotypeSet[protrusion]{basicmath} % disable protrusion for tt fonts
}{}
\makeatletter
\@ifundefined{KOMAClassName}{% if non-KOMA class
  \IfFileExists{parskip.sty}{%
    \usepackage{parskip}
  }{% else
    \setlength{\parindent}{0pt}
    \setlength{\parskip}{6pt plus 2pt minus 1pt}}
}{% if KOMA class
  \KOMAoptions{parskip=half}}
\makeatother
\usepackage{xcolor}
\IfFileExists{xurl.sty}{\usepackage{xurl}}{} % add URL line breaks if available
\IfFileExists{bookmark.sty}{\usepackage{bookmark}}{\usepackage{hyperref}}
\hypersetup{
  pdfauthor={Andreu Meca, Geraldo Gariza and Jan Leyva},
  hidelinks,
  pdfcreator={LaTeX via pandoc}}
\urlstyle{same} % disable monospaced font for URLs
\usepackage[margin=1in]{geometry}
\usepackage{color}
\usepackage{fancyvrb}
\newcommand{\VerbBar}{|}
\newcommand{\VERB}{\Verb[commandchars=\\\{\}]}
\DefineVerbatimEnvironment{Highlighting}{Verbatim}{commandchars=\\\{\}}
% Add ',fontsize=\small' for more characters per line
\usepackage{framed}
\definecolor{shadecolor}{RGB}{248,248,248}
\newenvironment{Shaded}{\begin{snugshade}}{\end{snugshade}}
\newcommand{\AlertTok}[1]{\textcolor[rgb]{0.94,0.16,0.16}{#1}}
\newcommand{\AnnotationTok}[1]{\textcolor[rgb]{0.56,0.35,0.01}{\textbf{\textit{#1}}}}
\newcommand{\AttributeTok}[1]{\textcolor[rgb]{0.77,0.63,0.00}{#1}}
\newcommand{\BaseNTok}[1]{\textcolor[rgb]{0.00,0.00,0.81}{#1}}
\newcommand{\BuiltInTok}[1]{#1}
\newcommand{\CharTok}[1]{\textcolor[rgb]{0.31,0.60,0.02}{#1}}
\newcommand{\CommentTok}[1]{\textcolor[rgb]{0.56,0.35,0.01}{\textit{#1}}}
\newcommand{\CommentVarTok}[1]{\textcolor[rgb]{0.56,0.35,0.01}{\textbf{\textit{#1}}}}
\newcommand{\ConstantTok}[1]{\textcolor[rgb]{0.00,0.00,0.00}{#1}}
\newcommand{\ControlFlowTok}[1]{\textcolor[rgb]{0.13,0.29,0.53}{\textbf{#1}}}
\newcommand{\DataTypeTok}[1]{\textcolor[rgb]{0.13,0.29,0.53}{#1}}
\newcommand{\DecValTok}[1]{\textcolor[rgb]{0.00,0.00,0.81}{#1}}
\newcommand{\DocumentationTok}[1]{\textcolor[rgb]{0.56,0.35,0.01}{\textbf{\textit{#1}}}}
\newcommand{\ErrorTok}[1]{\textcolor[rgb]{0.64,0.00,0.00}{\textbf{#1}}}
\newcommand{\ExtensionTok}[1]{#1}
\newcommand{\FloatTok}[1]{\textcolor[rgb]{0.00,0.00,0.81}{#1}}
\newcommand{\FunctionTok}[1]{\textcolor[rgb]{0.00,0.00,0.00}{#1}}
\newcommand{\ImportTok}[1]{#1}
\newcommand{\InformationTok}[1]{\textcolor[rgb]{0.56,0.35,0.01}{\textbf{\textit{#1}}}}
\newcommand{\KeywordTok}[1]{\textcolor[rgb]{0.13,0.29,0.53}{\textbf{#1}}}
\newcommand{\NormalTok}[1]{#1}
\newcommand{\OperatorTok}[1]{\textcolor[rgb]{0.81,0.36,0.00}{\textbf{#1}}}
\newcommand{\OtherTok}[1]{\textcolor[rgb]{0.56,0.35,0.01}{#1}}
\newcommand{\PreprocessorTok}[1]{\textcolor[rgb]{0.56,0.35,0.01}{\textit{#1}}}
\newcommand{\RegionMarkerTok}[1]{#1}
\newcommand{\SpecialCharTok}[1]{\textcolor[rgb]{0.00,0.00,0.00}{#1}}
\newcommand{\SpecialStringTok}[1]{\textcolor[rgb]{0.31,0.60,0.02}{#1}}
\newcommand{\StringTok}[1]{\textcolor[rgb]{0.31,0.60,0.02}{#1}}
\newcommand{\VariableTok}[1]{\textcolor[rgb]{0.00,0.00,0.00}{#1}}
\newcommand{\VerbatimStringTok}[1]{\textcolor[rgb]{0.31,0.60,0.02}{#1}}
\newcommand{\WarningTok}[1]{\textcolor[rgb]{0.56,0.35,0.01}{\textbf{\textit{#1}}}}
\usepackage{longtable,booktabs}
% Correct order of tables after \paragraph or \subparagraph
\usepackage{etoolbox}
\makeatletter
\patchcmd\longtable{\par}{\if@noskipsec\mbox{}\fi\par}{}{}
\makeatother
% Allow footnotes in longtable head/foot
\IfFileExists{footnotehyper.sty}{\usepackage{footnotehyper}}{\usepackage{footnote}}
\makesavenoteenv{longtable}
\usepackage{graphicx,grffile}
\makeatletter
\def\maxwidth{\ifdim\Gin@nat@width>\linewidth\linewidth\else\Gin@nat@width\fi}
\def\maxheight{\ifdim\Gin@nat@height>\textheight\textheight\else\Gin@nat@height\fi}
\makeatother
% Scale images if necessary, so that they will not overflow the page
% margins by default, and it is still possible to overwrite the defaults
% using explicit options in \includegraphics[width, height, ...]{}
\setkeys{Gin}{width=\maxwidth,height=\maxheight,keepaspectratio}
% Set default figure placement to htbp
\makeatletter
\def\fps@figure{htbp}
\makeatother
\setlength{\emergencystretch}{3em} % prevent overfull lines
\providecommand{\tightlist}{%
  \setlength{\itemsep}{0pt}\setlength{\parskip}{0pt}}
\setcounter{secnumdepth}{5}

\title{\textbf{Regression Trees Lab.}

Universitat Politècnica de Catalunya}
\author{Andreu Meca, Geraldo Gariza and Jan Leyva}
\date{6 de maig, 2021}

\begin{document}
\maketitle

{
\setcounter{tocdepth}{2}
\tableofcontents
}
\newpage

\begin{Shaded}
\begin{Highlighting}[]
\NormalTok{n <-}\StringTok{ }\KeywordTok{nrow}\NormalTok{(mydataset)}
\NormalTok{p <-}\StringTok{ }\KeywordTok{ncol}\NormalTok{(mydataset)}
\end{Highlighting}
\end{Shaded}

\hypertarget{description-data}{%
\section{Description data}\label{description-data}}

The Molecular Taxonomy of Breast Cancer International Consortium
(METABRIC) database is a Canada-UK Project which contains targeted
sequencing data of 1,980 primary breast cancer samples. Clinical and
genomic data was downloaded from cBioPortal. The dataset was collected
by Professor Carlos Caldas from Cambridge Research Institute and
Professor Sam Aparicio from the British Columbia Cancer Centre in Canada
and published on Nature Communications (Pereira et al., 2016)..

Metabric dataset integrates three types of data. Clinical data are in
columns 1 to 31, gene expression data are in columns 32 to 520, mutation
data are in columns 521 to 693. Gene expression values are normalized to
be z-score.

The data set has \textbf{1904} observations on \textbf{693} variables.

The aim of this study is to predict the \texttt{overall\_survival} that
specify if the patient is alive or dead. This will be done by three
models (tree, a random forest and gradient boosting in the packege
\texttt{adaboost}).

\begin{itemize}
\tightlist
\item
  Variability
\end{itemize}

\begin{Shaded}
\begin{Highlighting}[]
\KeywordTok{sort}\NormalTok{(}\KeywordTok{apply}\NormalTok{(mydataset[,}\DecValTok{32}\OperatorTok{:}\DecValTok{520}\NormalTok{] }\OperatorTok\StringTok{ }\KeywordTok{drop_na}\NormalTok{(), }\DecValTok{2}\NormalTok{, sd), }\DataTypeTok{decreasing =} \OtherTok{TRUE}\NormalTok{)[}\DecValTok{1}\OperatorTok{:}\DecValTok{6}\NormalTok{]}
\end{Highlighting}
\end{Shaded}

\begin{verbatim}
##    nfkb1    ptprm   cyp3a7     e2f1   pdgfra     ubr5 
## 1.000265 1.000265 1.000265 1.000264 1.000264 1.000264
\end{verbatim}

The top 3 variables with the most variability have 1.000265, which is
just a prove that the variables are normalized to be z-score

\begin{itemize}
\tightlist
\item
  Percentage missing values
\end{itemize}

\begin{Shaded}
\begin{Highlighting}[]
\NormalTok{n_miss <-}\StringTok{ }\KeywordTok{c}\NormalTok{()}
\ControlFlowTok{for}\NormalTok{(i }\ControlFlowTok{in} \DecValTok{1}\OperatorTok{:}\KeywordTok{ncol}\NormalTok{(mydataset))\{}
\NormalTok{  n_miss[i]<-}\KeywordTok{length}\NormalTok{(}\KeywordTok{which}\NormalTok{(}\KeywordTok{is.na}\NormalTok{(mydataset[,i])))}
\NormalTok{\}}
\NormalTok{knitr}\OperatorTok{::}\KeywordTok{kable}\NormalTok{(}\KeywordTok{t}\NormalTok{(n_miss[}\KeywordTok{which}\NormalTok{(n_miss }\OperatorTok{!=}\StringTok{ }\DecValTok{0}\NormalTok{)]), }\DataTypeTok{col.names =} \KeywordTok{which}\NormalTok{(n_miss }\OperatorTok{!=}\StringTok{ }\DecValTok{0}\NormalTok{))}
\end{Highlighting}
\end{Shaded}

\begin{longtable}[]{@{}rrrrrrrrrrrrrrrrr@{}}
\toprule
3 & 5 & 6 & 10 & 12 & 15 & 19 & 21 & 23 & 28 & 29 & 30 & 31 & 679 & 689
& 691 & 693\tabularnewline
\midrule
\endhead
22 & 15 & 54 & 30 & 72 & 15 & 106 & 45 & 15 & 204 & 20 & 501 & 1 & 7 & 2
& 2 & 1\tabularnewline
\bottomrule
\end{longtable}

Most of the missing values are on the first columns of the dataset, from
1 to 32. On the figure 1 is possible see the missing percentage for the
32 first columns.

\begin{Shaded}
\begin{Highlighting}[]
\KeywordTok{cat}\NormalTok{(}\StringTok{"The percentage of NA is:"}\NormalTok{, }\KeywordTok{round}\NormalTok{((}\KeywordTok{length}\NormalTok{(}\KeywordTok{which}\NormalTok{(n_miss }\OperatorTok{!=}\StringTok{ }\DecValTok{0}\NormalTok{))}\OperatorTok{/}\KeywordTok{ncol}\NormalTok{(mydataset)}\OperatorTok{*}\DecValTok{100}\NormalTok{),}\DecValTok{2}\NormalTok{),}\StringTok{"%"}\NormalTok{)}
\end{Highlighting}
\end{Shaded}

\begin{verbatim}
## The percentage of NA is: 2.45 %
\end{verbatim}

\begin{Shaded}
\begin{Highlighting}[]
\KeywordTok{vis_miss}\NormalTok{(mydataset[,}\DecValTok{1}\OperatorTok{:}\DecValTok{32}\NormalTok{])}
\end{Highlighting}
\end{Shaded}

\begin{figure}
\centering
\includegraphics{Assignment-Tree_files/figure-latex/unnamed-chunk-4-1.pdf}
\caption{Missing values Col 1 to 32}
\end{figure}

\hypertarget{preprocess}{%
\section{Preprocess}\label{preprocess}}

In this model is not needed a data preprocess.

\hypertarget{partition-of-data}{%
\section{Partition of data}\label{partition-of-data}}

In order to properly evaluate the performance of a model, we must
estimate the error rather than simply computing the training error. We
split the observations into a training set and a test set, build the
model using the training set, and evaluate its performance on the test
data.

\begin{Shaded}
\begin{Highlighting}[]
\KeywordTok{set.seed}\NormalTok{(params}\OperatorTok{$}\NormalTok{seed)}
\NormalTok{data <-mydataset[, }\DecValTok{32}\OperatorTok{:}\DecValTok{520}\NormalTok{]}
\NormalTok{data}\OperatorTok{$}\NormalTok{overall_survival <-}\StringTok{ }\NormalTok{mydataset}\OperatorTok{$}\NormalTok{overall_survival}
\NormalTok{trainIndex <-}\StringTok{ }\KeywordTok{createDataPartition}\NormalTok{(data}\OperatorTok{$}\NormalTok{brca1, }\DataTypeTok{p =}\NormalTok{ params}\OperatorTok{$}\NormalTok{partition, }\DataTypeTok{list =} \OtherTok{FALSE}\NormalTok{, }\DataTypeTok{times =} \DecValTok{1}\NormalTok{)}
\NormalTok{train <-}\StringTok{ }\NormalTok{data[ trainIndex,]}
\NormalTok{test  <-}\StringTok{ }\NormalTok{data[}\OperatorTok{-}\NormalTok{trainIndex,]}
\end{Highlighting}
\end{Shaded}

\begin{Shaded}
\begin{Highlighting}[]
\KeywordTok{set.seed}\NormalTok{(params}\OperatorTok{$}\NormalTok{seed)}
\NormalTok{data_split <-}\StringTok{ }\KeywordTok{initial_split}\NormalTok{(data, }\DataTypeTok{prop =}\NormalTok{ params}\OperatorTok{$}\NormalTok{partition)}
\NormalTok{train <-}\StringTok{ }\KeywordTok{training}\NormalTok{(data_split)}
\NormalTok{test  <-}\StringTok{ }\KeywordTok{testing}\NormalTok{(data_split)}
\end{Highlighting}
\end{Shaded}

The train set has 1270 observations and the test set has 634.

\begin{enumerate}
\def\labelenumi{\arabic{enumi}.}
\setcounter{enumi}{2}
\tightlist
\item
  Fit a pruned single tree model to predict the overall\_survival.
  Assess the performance of the tree by using suitable metrics.
  overall\_survival is a target variable whether the patient is alive or
  dead. Additionally, you can adjust another model that adds some
  clinical variable that you consider relevant to the prediction.
\end{enumerate}

\begin{Shaded}
\begin{Highlighting}[]
\NormalTok{fit <-}\StringTok{ }\KeywordTok{rpart}\NormalTok{(overall_survival}\OperatorTok{~}\NormalTok{., }\DataTypeTok{method=}\StringTok{"class"}\NormalTok{, }\DataTypeTok{data=}\NormalTok{train)}
\NormalTok{fit_p <-}\StringTok{ }\KeywordTok{prune}\NormalTok{(fit, }\DataTypeTok{cp =}\NormalTok{ fit}\OperatorTok{$}\NormalTok{cptable[}\KeywordTok{which.min}\NormalTok{(fit}\OperatorTok{$}\NormalTok{cptable[,}\StringTok{"xerror"}\NormalTok{]),}\StringTok{"CP"}\NormalTok{])}
\NormalTok{res_fit <-}\StringTok{ }\KeywordTok{predict}\NormalTok{(fit_p, }\DataTypeTok{newdata =}\NormalTok{ test, }\DataTypeTok{type =} \StringTok{"class"}\NormalTok{)}
\end{Highlighting}
\end{Shaded}

The package rpart fits a simple classification trees, which is stored in
the variable fit and then it is pruned to not fall in an overfitting
context. The prune is done minimizing the error given by the xerror
parameter, computed with the cross-validation implementation.

Then we predict using the test set and store it in the res\_fit object.

Now, we compute the confusion matrix and the model accuracy.

\begin{Shaded}
\begin{Highlighting}[]
\KeywordTok{confusion_matrix}\NormalTok{(test}\OperatorTok{$}\NormalTok{overall_survival, }\KeywordTok{as.numeric}\NormalTok{(}\KeywordTok{as.character}\NormalTok{(res_fit)))}
\end{Highlighting}
\end{Shaded}

\begin{longtable}[]{@{}lrr@{}}
\toprule
& Actual\_0 & Actual\_1\tabularnewline
\midrule
\endhead
Predicted\_0 & 273 & 89\tabularnewline
Predicted\_1 & 162 & 110\tabularnewline
\bottomrule
\end{longtable}

\begin{Shaded}
\begin{Highlighting}[]
\KeywordTok{accuracy}\NormalTok{(test}\OperatorTok{$}\NormalTok{overall_survival, }\KeywordTok{as.numeric}\NormalTok{(}\KeywordTok{as.character}\NormalTok{(res_fit)))}
\end{Highlighting}
\end{Shaded}

\begin{verbatim}
## Accuracy: 0.6041009
\end{verbatim}

\begin{enumerate}
\def\labelenumi{\arabic{enumi}.}
\setcounter{enumi}{3}
\tightlist
\item
  Fit a Random Forest (RF) classifier to predict the overall\_survival.
  Tune the pa- rameters: number of trees and number of variables per
  node, by implementing a grid search procedure. Assess the performance
  of RF using suitable metrics. Determine which variables are the most
  relevant in the overall\_survival prediction.
\end{enumerate}

\begin{Shaded}
\begin{Highlighting}[]
\CommentTok{# colnames(train)}

\NormalTok{bag.survival=}\KeywordTok{randomForest}\NormalTok{(}\DataTypeTok{x=}\NormalTok{mydataset[,}\DecValTok{32}\OperatorTok{:}\DecValTok{520}\NormalTok{],}\DataTypeTok{y=}\NormalTok{mydataset}\OperatorTok{$}\NormalTok{overall_survival,}\DataTypeTok{data=}\NormalTok{mydataset}
\NormalTok{                          ,}\DataTypeTok{subset=}\NormalTok{train,}\DataTypeTok{importance=}\OtherTok{TRUE}\NormalTok{, }\DataTypeTok{proximity =} \OtherTok{TRUE}\NormalTok{)}
\end{Highlighting}
\end{Shaded}

\begin{verbatim}
## Warning in randomForest.default(x = mydataset[, 32:520], y =
## mydataset$overall_survival, : The response has five or fewer unique values. Are
## you sure you want to do regression?
\end{verbatim}

\begin{Shaded}
\begin{Highlighting}[]
\NormalTok{bag.survival}
\end{Highlighting}
\end{Shaded}

\begin{verbatim}
## 
## Call:
##  randomForest(x = mydataset[, 32:520], y = mydataset$overall_survival,      importance = TRUE, proximity = TRUE, data = mydataset, subset = train) 
##                Type of random forest: regression
##                      Number of trees: 500
## No. of variables tried at each split: 163
## 
##           Mean of squared residuals: 0.2233381
##                     % Var explained: 8.36
\end{verbatim}

\begin{Shaded}
\begin{Highlighting}[]
\NormalTok{OBB.error<-bag.survival}\OperatorTok{$}\NormalTok{err.rate[}\KeywordTok{nrow}\NormalTok{(bag.survival}\OperatorTok{$}\NormalTok{err.rate),}\DecValTok{1}\NormalTok{]}
\end{Highlighting}
\end{Shaded}

We see the error rate is with the number of trees and the nº of
variables tried at each split = 22. We can plot the forest in order to
see in that number of trees is the optimum.

\begin{Shaded}
\begin{Highlighting}[]
\CommentTok{#We use the matrix of errors to make a visualization}
\NormalTok{ntrees<-}\KeywordTok{nrow}\NormalTok{(bag.survival}\OperatorTok{$}\NormalTok{err.rate)}

\NormalTok{oob.error.data<-}\StringTok{ }\KeywordTok{data.frame}\NormalTok{(}
  \DataTypeTok{Trees=}\KeywordTok{rep}\NormalTok{(}\DecValTok{1}\OperatorTok{:}\NormalTok{ntrees, }\DataTypeTok{times=}\DecValTok{3}\NormalTok{),}
  \DataTypeTok{Type=} \KeywordTok{rep}\NormalTok{(}\KeywordTok{c}\NormalTok{(}\StringTok{"OOB"}\NormalTok{, }\StringTok{"0"}\NormalTok{, }\StringTok{"1"}\NormalTok{), }\DataTypeTok{each=}\NormalTok{ntrees),}
  \DataTypeTok{Error=}\KeywordTok{c}\NormalTok{(bag.survival}\OperatorTok{$}\NormalTok{err.rate[,}\StringTok{"OOB"}\NormalTok{], }
\NormalTok{          bag.survival}\OperatorTok{$}\NormalTok{err.rate[,}\StringTok{"0"}\NormalTok{], }
\NormalTok{          bag.survival}\OperatorTok{$}\NormalTok{err.rate[,}\StringTok{"1"}\NormalTok{]))}
\end{Highlighting}
\end{Shaded}

\begin{verbatim}
## Error in 1:ntrees: argument of length 0
\end{verbatim}

\begin{Shaded}
\begin{Highlighting}[]
\KeywordTok{ggplot}\NormalTok{(}\DataTypeTok{data=}\NormalTok{oob.error.data, }\KeywordTok{aes}\NormalTok{(}\DataTypeTok{x=}\NormalTok{Trees, }\DataTypeTok{y=}\NormalTok{Error)) }\OperatorTok{+}
\StringTok{  }\KeywordTok{geom_line}\NormalTok{(}\KeywordTok{aes}\NormalTok{(}\DataTypeTok{color=}\NormalTok{Type))}
\end{Highlighting}
\end{Shaded}

\begin{verbatim}
## Error in ggplot(data = oob.error.data, aes(x = Trees, y = Error)): object 'oob.error.data' not found
\end{verbatim}

Now, let's try if with a higher number of trees we obtain a better
performance:

\begin{Shaded}
\begin{Highlighting}[]
\NormalTok{bag.survival2=}\KeywordTok{randomForest}\NormalTok{(}\DataTypeTok{x=}\NormalTok{mydataset[,}\DecValTok{32}\OperatorTok{:}\DecValTok{520}\NormalTok{],}\DataTypeTok{y=}\NormalTok{mydataset}\OperatorTok{$}\NormalTok{overall_survival,}\DataTypeTok{data=}\NormalTok{mydataset}
\NormalTok{                          ,}\DataTypeTok{subset=}\NormalTok{train,}\DataTypeTok{importance=}\OtherTok{TRUE}\NormalTok{,}\DataTypeTok{ntree =} \DecValTok{1000}\NormalTok{ ,}\DataTypeTok{proximity =} \OtherTok{TRUE}\NormalTok{)}
\end{Highlighting}
\end{Shaded}

\begin{verbatim}
## Warning in randomForest.default(x = mydataset[, 32:520], y =
## mydataset$overall_survival, : The response has five or fewer unique values. Are
## you sure you want to do regression?
\end{verbatim}

\begin{Shaded}
\begin{Highlighting}[]
\NormalTok{bag.survival2}
\end{Highlighting}
\end{Shaded}

\begin{verbatim}
## 
## Call:
##  randomForest(x = mydataset[, 32:520], y = mydataset$overall_survival,      ntree = 1000, importance = TRUE, proximity = TRUE, data = mydataset,      subset = train) 
##                Type of random forest: regression
##                      Number of trees: 1000
## No. of variables tried at each split: 163
## 
##           Mean of squared residuals: 0.2217049
##                     % Var explained: 9.03
\end{verbatim}

\begin{Shaded}
\begin{Highlighting}[]
\NormalTok{OBB.error2<-bag.survival2}\OperatorTok{$}\NormalTok{err.rate[}\KeywordTok{nrow}\NormalTok{(bag.survival}\OperatorTok{$}\NormalTok{err.rate),}\DecValTok{1}\NormalTok{]}
\end{Highlighting}
\end{Shaded}

We see the error rate is with the number of trees and the nº of
variables tried at each split = 22. We cannot guarantee a higher
performance, thus we remain with .

\begin{Shaded}
\begin{Highlighting}[]
\NormalTok{ntrees2<-}\KeywordTok{nrow}\NormalTok{(bag.survival2}\OperatorTok{$}\NormalTok{err.rate)}

\NormalTok{oob.error.data2<-}\StringTok{ }\KeywordTok{data.frame}\NormalTok{(}
  \DataTypeTok{Trees=}\KeywordTok{rep}\NormalTok{(}\DecValTok{1}\OperatorTok{:}\NormalTok{ntrees2, }\DataTypeTok{times=}\DecValTok{3}\NormalTok{),}
  \DataTypeTok{Type=} \KeywordTok{rep}\NormalTok{(}\KeywordTok{c}\NormalTok{(}\StringTok{"OOB"}\NormalTok{, }\StringTok{"0"}\NormalTok{, }\StringTok{"1"}\NormalTok{), }\DataTypeTok{each=}\NormalTok{ntrees2),}
  \DataTypeTok{Error=}\KeywordTok{c}\NormalTok{(bag.survival2}\OperatorTok{$}\NormalTok{err.rate[,}\StringTok{"OOB"}\NormalTok{], }
\NormalTok{          bag.survival2}\OperatorTok{$}\NormalTok{err.rate[,}\StringTok{"0"}\NormalTok{], }
\NormalTok{          bag.survival2}\OperatorTok{$}\NormalTok{err.rate[,}\StringTok{"1"}\NormalTok{]))}
\end{Highlighting}
\end{Shaded}

\begin{verbatim}
## Error in 1:ntrees2: argument of length 0
\end{verbatim}

\begin{Shaded}
\begin{Highlighting}[]
\KeywordTok{ggplot}\NormalTok{(}\DataTypeTok{data=}\NormalTok{oob.error.data2, }\KeywordTok{aes}\NormalTok{(}\DataTypeTok{x=}\NormalTok{Trees, }\DataTypeTok{y=}\NormalTok{Error)) }\OperatorTok{+}
\StringTok{  }\KeywordTok{geom_line}\NormalTok{(}\KeywordTok{aes}\NormalTok{(}\DataTypeTok{color=}\NormalTok{Type))}
\end{Highlighting}
\end{Shaded}

\begin{verbatim}
## Error in ggplot(data = oob.error.data2, aes(x = Trees, y = Error)): object 'oob.error.data2' not found
\end{verbatim}

The plot represents stability with more than 500 trees.

Now, let's try to change the nº of variables tried each split:

\begin{Shaded}
\begin{Highlighting}[]
\KeywordTok{set.seed}\NormalTok{(}\DecValTok{1234}\NormalTok{)}
\NormalTok{oob.values <-}\StringTok{ }\KeywordTok{vector}\NormalTok{(}\DataTypeTok{length=}\DecValTok{25}\NormalTok{)}
\ControlFlowTok{for}\NormalTok{(i }\ControlFlowTok{in} \DecValTok{20}\OperatorTok{:}\DecValTok{25}\NormalTok{) \{}
\NormalTok{  temp.model <-}\StringTok{ }\KeywordTok{randomForest}\NormalTok{(}\DataTypeTok{x=}\NormalTok{mydataset[,}\DecValTok{32}\OperatorTok{:}\DecValTok{520}\NormalTok{],}\DataTypeTok{y=}\NormalTok{mydataset}\OperatorTok{$}\NormalTok{overall_survival,}\DataTypeTok{data=}\NormalTok{mydataset}
\NormalTok{                             ,}\DataTypeTok{subset=}\NormalTok{train,}\DataTypeTok{importance=}\OtherTok{TRUE}\NormalTok{,}\DataTypeTok{ntree =} \DecValTok{500}\NormalTok{ ,}\DataTypeTok{proximity =} \OtherTok{TRUE}\NormalTok{)}
  
\NormalTok{  oob.values[i] <-}\StringTok{ }\NormalTok{temp.model}\OperatorTok{$}\NormalTok{err.rate[}\KeywordTok{nrow}\NormalTok{(temp.model}\OperatorTok{$}\NormalTok{err.rate),}\DecValTok{1}\NormalTok{]}
\NormalTok{\}}
\end{Highlighting}
\end{Shaded}

\begin{verbatim}
## Warning in randomForest.default(x = mydataset[, 32:520], y =
## mydataset$overall_survival, : The response has five or fewer unique values. Are
## you sure you want to do regression?
\end{verbatim}

\begin{verbatim}
## Error in oob.values[i] <- temp.model$err.rate[nrow(temp.model$err.rate), : replacement has length zero
\end{verbatim}

We observe the minimum at position 24º therefore, we use the model as
default:

\begin{Shaded}
\begin{Highlighting}[]
\NormalTok{oob.values}
\end{Highlighting}
\end{Shaded}

\begin{verbatim}
##  [1] FALSE FALSE FALSE FALSE FALSE FALSE FALSE FALSE FALSE FALSE FALSE FALSE
## [13] FALSE FALSE FALSE FALSE FALSE FALSE FALSE FALSE FALSE FALSE FALSE FALSE
## [25] FALSE
\end{verbatim}

\begin{Shaded}
\begin{Highlighting}[]
\CommentTok{## find the minimum error}
\NormalTok{oob.values<-}\DecValTok{1}\OperatorTok{-}\NormalTok{oob.values}

\CommentTok{## find the optimal value for mtry...}
\KeywordTok{which}\NormalTok{(oob.values }\OperatorTok{==}\StringTok{ }\KeywordTok{min}\NormalTok{(oob.values))}
\end{Highlighting}
\end{Shaded}

\begin{verbatim}
##  [1]  1  2  3  4  5  6  7  8  9 10 11 12 13 14 15 16 17 18 19 20 21 22 23 24 25
\end{verbatim}

\begin{Shaded}
\begin{Highlighting}[]
\CommentTok{## create a model for proximities using the best value for mtry}
\NormalTok{model <-}\StringTok{ }\KeywordTok{randomForest}\NormalTok{(}\DataTypeTok{x=}\NormalTok{mydataset[,}\DecValTok{32}\OperatorTok{:}\DecValTok{520}\NormalTok{],}\DataTypeTok{y=}\NormalTok{mydataset}\OperatorTok{$}\NormalTok{overall_survival,}\DataTypeTok{data=}\NormalTok{mydataset}
\NormalTok{                      ,}\DataTypeTok{subset=}\NormalTok{train,}\DataTypeTok{importance=}\OtherTok{TRUE}\NormalTok{,}\DataTypeTok{ntree =} \DecValTok{500}\NormalTok{ ,}\DataTypeTok{proximity =} \OtherTok{TRUE}
\NormalTok{                      ,}\DataTypeTok{mtry=}\KeywordTok{which}\NormalTok{(oob.values }\OperatorTok{==}\StringTok{ }\KeywordTok{min}\NormalTok{(oob.values)))}
\end{Highlighting}
\end{Shaded}

\begin{verbatim}
## Warning in randomForest.default(x = mydataset[, 32:520], y =
## mydataset$overall_survival, : The response has five or fewer unique values. Are
## you sure you want to do regression?
\end{verbatim}

We are going to calculate the confusion matrix for the test sample:

\begin{Shaded}
\begin{Highlighting}[]
\NormalTok{model}
\end{Highlighting}
\end{Shaded}

\begin{verbatim}
## 
## Call:
##  randomForest(x = mydataset[, 32:520], y = mydataset$overall_survival,      ntree = 500, mtry = which(oob.values == min(oob.values)),      importance = TRUE, proximity = TRUE, data = mydataset, subset = train) 
##                Type of random forest: regression
##                      Number of trees: 500
## No. of variables tried at each split: 1
## 
##           Mean of squared residuals: 0.2286382
##                     % Var explained: 6.18
\end{verbatim}

\begin{Shaded}
\begin{Highlighting}[]
\NormalTok{yhat.bag =}\StringTok{ }\KeywordTok{predict}\NormalTok{(model,}\DataTypeTok{newdata=}\NormalTok{test[,}\DecValTok{32}\OperatorTok{:}\DecValTok{520}\NormalTok{])}
\end{Highlighting}
\end{Shaded}

\begin{verbatim}
## Error: Can't subset columns that don't exist.
## x Locations 491, 492, 493, 494, 495, etc. don't exist.
## i There are only 490 columns.
\end{verbatim}

\begin{Shaded}
\begin{Highlighting}[]
\NormalTok{conf.matrix.rf<-}\KeywordTok{confusionMatrix}\NormalTok{(}\KeywordTok{as.factor}\NormalTok{(yhat.bag), }
                                \KeywordTok{as.factor}\NormalTok{(test}\OperatorTok{$}\NormalTok{overall_survival))}
\end{Highlighting}
\end{Shaded}

\begin{verbatim}
## Error in is.factor(x): object 'yhat.bag' not found
\end{verbatim}

\begin{Shaded}
\begin{Highlighting}[]
\NormalTok{conf.matrix.rf}\OperatorTok{$}\NormalTok{table; conf.matrix.rf}\OperatorTok{$}\NormalTok{overall[}\DecValTok{1}\NormalTok{]}
\end{Highlighting}
\end{Shaded}

\begin{verbatim}
## Error in eval(expr, envir, enclos): object 'conf.matrix.rf' not found
\end{verbatim}

\begin{verbatim}
## Error in eval(expr, envir, enclos): object 'conf.matrix.rf' not found
\end{verbatim}

We use the varImplot function to see the importance of the variables:

\begin{Shaded}
\begin{Highlighting}[]
\KeywordTok{varImpPlot}\NormalTok{(model)}
\end{Highlighting}
\end{Shaded}

\includegraphics{Assignment-Tree_files/figure-latex/unnamed-chunk-16-1.pdf}

\begin{enumerate}
\def\labelenumi{\arabic{enumi}.}
\setcounter{enumi}{4}
\tightlist
\item
  Apply the gradient boosting algorithm with adaboost specification:
\end{enumerate}

5.1. Using stumps as classification trees for \texttt{overall\_survival}
prediction, compute the misclassification rates of both the learning set
and the test set across 2,000 iterations of gbm. Represent graphically
the error as a function of the number of boosting iterations.

\begin{Shaded}
\begin{Highlighting}[]
\NormalTok{boost_model=}\KeywordTok{gbm}\NormalTok{(overall_survival}\OperatorTok{~}\NormalTok{.,}
                \DataTypeTok{data =}\NormalTok{ train, }
                \DataTypeTok{distribution=}\StringTok{"adaboost"}\NormalTok{, }
                \DataTypeTok{n.trees=}\DecValTok{2000}\NormalTok{, }\DataTypeTok{interaction.depth=}\DecValTok{1}\NormalTok{, }
                \DataTypeTok{cv.folds =} \DecValTok{3}\NormalTok{)}
\end{Highlighting}
\end{Shaded}

The package \texttt{gbm} have the distribution \texttt{adaboost} that
perform the adaboost algorithm. In this case is a model with a 2000
trees of one deep each one (stumps). Also, it is used a cv.folds = 3.

\begin{itemize}
\tightlist
\item
  Represent graphically the error as a function of the number of
  boosting iterations.
\end{itemize}

\begin{Shaded}
\begin{Highlighting}[]
\NormalTok{(perf<-}\KeywordTok{gbm.perf}\NormalTok{(boost_model))}
\end{Highlighting}
\end{Shaded}

\includegraphics{Assignment-Tree_files/figure-latex/unnamed-chunk-18-1.pdf}

\begin{verbatim}
## [1] 41
\end{verbatim}

The gbm.perf return a plot where is specify the number of iterations
need it in order to get a good performance. In this case the number of
iterations for got a similar results that done with 2000 trees is 41. It
means that if we perform again the \texttt{adaboost} but this time with
\texttt{n.trees=} 41 we should get a similar results that the obtaineds
with 2000.

\begin{Shaded}
\begin{Highlighting}[]
\NormalTok{boost_predict_train=}\KeywordTok{predict}\NormalTok{(boost_model,}\DataTypeTok{newdata=}\NormalTok{train,}
                         \DataTypeTok{n.trees=}\DecValTok{2000}\NormalTok{, }\DataTypeTok{type =} \StringTok{"response"}\NormalTok{)}
\NormalTok{conf_matr_train<-}\KeywordTok{confusionMatrix}\NormalTok{(}\KeywordTok{as.factor}\NormalTok{(}\KeywordTok{round}\NormalTok{(boost_predict_train)), }
                \KeywordTok{as.factor}\NormalTok{(train}\OperatorTok{$}\NormalTok{overall_survival))}
\NormalTok{conf_matr_train}\OperatorTok{$}\NormalTok{table; conf_matr_train}\OperatorTok{$}\NormalTok{overall[}\DecValTok{1}\NormalTok{]}
\end{Highlighting}
\end{Shaded}

\begin{verbatim}
##           Reference
## Prediction   0   1
##          0 741   0
##          1   0 529
\end{verbatim}

\begin{verbatim}
## Accuracy 
##        1
\end{verbatim}

As expected the performance with the same data that the algorithm is
trained the performance is perfect with a 0 error rate. This is one of
the problems of adaboost that tends to overfit the data, for this reason
is used a split data in train and test.

\begin{Shaded}
\begin{Highlighting}[]
\NormalTok{boost_predict=}\KeywordTok{predict}\NormalTok{(boost_model,}\DataTypeTok{newdata=}\NormalTok{test,}
                         \DataTypeTok{n.trees=}\DecValTok{2000}\NormalTok{, }\DataTypeTok{type =} \StringTok{"response"}\NormalTok{)}
\NormalTok{conf_matr<-}\KeywordTok{confusionMatrix}\NormalTok{(}\KeywordTok{as.factor}\NormalTok{(}\KeywordTok{round}\NormalTok{(boost_predict)), }
                \KeywordTok{as.factor}\NormalTok{(test}\OperatorTok{$}\NormalTok{overall_survival))}
\NormalTok{conf_matr}\OperatorTok{$}\NormalTok{table; conf_matr}\OperatorTok{$}\NormalTok{overall[}\DecValTok{1}\NormalTok{]}
\end{Highlighting}
\end{Shaded}

\begin{verbatim}
##           Reference
## Prediction   0   1
##          0 266 151
##          1  96 121
\end{verbatim}

\begin{verbatim}
##  Accuracy 
## 0.6104101
\end{verbatim}

The performance when is predicted the test data set is 0.6104101.

\hypertarget{prediction}{%
\section{Prediction}\label{prediction}}

5.2. Compare the test-set misclassification rates attained by different
ensemble classifiers based on trees with maximum depth: stumps, 4-node
trees, 8-node trees, and 16-node trees.

\begin{Shaded}
\begin{Highlighting}[]
\NormalTok{boost_model_}\DecValTok{4}\NormalTok{=}\KeywordTok{gbm}\NormalTok{(overall_survival}\OperatorTok{~}\NormalTok{.,}\DataTypeTok{data=}\NormalTok{ train, }
                  \DataTypeTok{distribution=}\StringTok{"adaboost"}\NormalTok{, }\DataTypeTok{n.trees=}\DecValTok{2000}\NormalTok{, }
                  \DataTypeTok{interaction.depth=}\DecValTok{4}\NormalTok{, }\DataTypeTok{cv.folds =} \DecValTok{3}\NormalTok{)}
\NormalTok{boost_predict_}\DecValTok{4}\NormalTok{ =}\KeywordTok{predict}\NormalTok{(boost_model_}\DecValTok{4}\NormalTok{,}\DataTypeTok{newdata=}\NormalTok{test,}
                         \DataTypeTok{n.trees=}\DecValTok{2000}\NormalTok{, }\DataTypeTok{type =} \StringTok{"response"}\NormalTok{)}
\NormalTok{conf_matr_}\DecValTok{4}\NormalTok{ <-}\StringTok{ }\KeywordTok{confusionMatrix}\NormalTok{(}\KeywordTok{as.factor}\NormalTok{(}\KeywordTok{round}\NormalTok{(boost_predict_}\DecValTok{4}\NormalTok{)), }
                \KeywordTok{as.factor}\NormalTok{(test}\OperatorTok{$}\NormalTok{overall_survival))}
\NormalTok{conf_matr_}\DecValTok{4}\OperatorTok{$}\NormalTok{table; conf_matr_}\DecValTok{4}\OperatorTok{$}\NormalTok{overall[}\DecValTok{1}\NormalTok{]}
\end{Highlighting}
\end{Shaded}

\begin{verbatim}
##           Reference
## Prediction   0   1
##          0 269 141
##          1  93 131
\end{verbatim}

\begin{verbatim}
##  Accuracy 
## 0.6309148
\end{verbatim}

\begin{Shaded}
\begin{Highlighting}[]
\KeywordTok{mean}\NormalTok{(}\KeywordTok{round}\NormalTok{(boost_predict_}\DecValTok{4}\NormalTok{) }\OperatorTok{!=}\StringTok{ }\NormalTok{test}\OperatorTok{$}\NormalTok{overall_survival)}
\end{Highlighting}
\end{Shaded}

\begin{verbatim}
## [1] 0.3690852
\end{verbatim}

\begin{Shaded}
\begin{Highlighting}[]
\NormalTok{boost_model_}\DecValTok{8}\NormalTok{=}\KeywordTok{gbm}\NormalTok{(train}\OperatorTok{$}\NormalTok{overall_survival}\OperatorTok{~}\NormalTok{.,}\DataTypeTok{data=}\NormalTok{ (train), }\DataTypeTok{distribution=}\StringTok{"adaboost"}\NormalTok{, }\DataTypeTok{n.trees=}\DecValTok{2000}\NormalTok{, }\DataTypeTok{interaction.depth=}\DecValTok{8}\NormalTok{, }\DataTypeTok{cv.folds =} \DecValTok{3}\NormalTok{)}
\NormalTok{boost_predict_}\DecValTok{8}\NormalTok{ =}\KeywordTok{predict}\NormalTok{(boost_model_}\DecValTok{8}\NormalTok{,}\DataTypeTok{newdata=}\NormalTok{test,}
                         \DataTypeTok{n.trees=}\DecValTok{2000}\NormalTok{, }\DataTypeTok{type =} \StringTok{"response"}\NormalTok{)}
\NormalTok{conf_matr_}\DecValTok{8}\NormalTok{ <-}\StringTok{ }\KeywordTok{confusionMatrix}\NormalTok{(}\KeywordTok{as.factor}\NormalTok{(}\KeywordTok{round}\NormalTok{(boost_predict_}\DecValTok{8}\NormalTok{)), }
                \KeywordTok{as.factor}\NormalTok{(test}\OperatorTok{$}\NormalTok{overall_survival))}
\NormalTok{conf_matr_}\DecValTok{8}\OperatorTok{$}\NormalTok{table; conf_matr_}\DecValTok{8}\OperatorTok{$}\NormalTok{overall[}\DecValTok{1}\NormalTok{]}
\end{Highlighting}
\end{Shaded}

\begin{verbatim}
##           Reference
## Prediction   0   1
##          0 275 142
##          1  87 130
\end{verbatim}

\begin{verbatim}
##  Accuracy 
## 0.6388013
\end{verbatim}

\begin{Shaded}
\begin{Highlighting}[]
\KeywordTok{mean}\NormalTok{(}\KeywordTok{round}\NormalTok{(boost_predict_}\DecValTok{8}\NormalTok{) }\OperatorTok{!=}\StringTok{ }\NormalTok{test}\OperatorTok{$}\NormalTok{overall_survival)}
\end{Highlighting}
\end{Shaded}

\begin{verbatim}
## [1] 0.3611987
\end{verbatim}

\begin{Shaded}
\begin{Highlighting}[]
\CommentTok{#summary(boost_model)}
\end{Highlighting}
\end{Shaded}

\begin{Shaded}
\begin{Highlighting}[]
\NormalTok{boost_model_}\DecValTok{16}\NormalTok{=}\KeywordTok{gbm}\NormalTok{(train}\OperatorTok{$}\NormalTok{overall_survival}\OperatorTok{~}\NormalTok{.,}\DataTypeTok{data=}\NormalTok{ (train[, }\DecValTok{32}\OperatorTok{:}\DecValTok{520}\NormalTok{]), }\DataTypeTok{distribution=}\StringTok{"adaboost"}\NormalTok{, }\DataTypeTok{n.trees=}\DecValTok{2000}\NormalTok{, }\DataTypeTok{interaction.depth=}\DecValTok{16}\NormalTok{, }\DataTypeTok{cv.folds =} \DecValTok{3}\NormalTok{)}
\end{Highlighting}
\end{Shaded}

\begin{verbatim}
## Error: Can't subset columns that don't exist.
## x Locations 491, 492, 493, 494, 495, etc. don't exist.
## i There are only 490 columns.
\end{verbatim}

\begin{Shaded}
\begin{Highlighting}[]
\NormalTok{boost_predict_}\DecValTok{16}\NormalTok{ =}\KeywordTok{predict}\NormalTok{(boost_model_}\DecValTok{16}\NormalTok{,}\DataTypeTok{newdata=}\NormalTok{test,}
                         \DataTypeTok{n.trees=}\DecValTok{2000}\NormalTok{, }\DataTypeTok{type =} \StringTok{"response"}\NormalTok{)}
\end{Highlighting}
\end{Shaded}

\begin{verbatim}
## Error in predict(boost_model_16, newdata = test, n.trees = 2000, type = "response"): object 'boost_model_16' not found
\end{verbatim}

\begin{Shaded}
\begin{Highlighting}[]
\NormalTok{conf_matr_}\DecValTok{16}\NormalTok{ <-}\StringTok{ }\KeywordTok{confusionMatrix}\NormalTok{(}\KeywordTok{as.factor}\NormalTok{(}\KeywordTok{round}\NormalTok{(boost_predict_}\DecValTok{16}\NormalTok{)), }
                \KeywordTok{as.factor}\NormalTok{(test}\OperatorTok{$}\NormalTok{overall_survival))}
\end{Highlighting}
\end{Shaded}

\begin{verbatim}
## Error in is.factor(x): object 'boost_predict_16' not found
\end{verbatim}

\begin{Shaded}
\begin{Highlighting}[]
\NormalTok{conf_matr_}\DecValTok{16}\OperatorTok{$}\NormalTok{table; conf_matr_}\DecValTok{16}\OperatorTok{$}\NormalTok{overall[}\DecValTok{1}\NormalTok{]}
\end{Highlighting}
\end{Shaded}

\begin{verbatim}
## Error in eval(expr, envir, enclos): object 'conf_matr_16' not found
\end{verbatim}

\begin{verbatim}
## Error in eval(expr, envir, enclos): object 'conf_matr_16' not found
\end{verbatim}

\begin{Shaded}
\begin{Highlighting}[]
\KeywordTok{mean}\NormalTok{(}\KeywordTok{round}\NormalTok{(boost_predict_}\DecValTok{16}\NormalTok{) }\OperatorTok{!=}\StringTok{ }\NormalTok{test}\OperatorTok{$}\NormalTok{overall_survival)}
\end{Highlighting}
\end{Shaded}

\begin{verbatim}
## Error in mean(round(boost_predict_16) != test$overall_survival): object 'boost_predict_16' not found
\end{verbatim}

\textbf{Summary table}

\begin{Shaded}
\begin{Highlighting}[]
\KeywordTok{kable}\NormalTok{(}\KeywordTok{matrix}\NormalTok{(}\KeywordTok{c}\NormalTok{(}\StringTok{"Stump"}\NormalTok{, }\StringTok{"4-node trees"}\NormalTok{, }\StringTok{"8-node trees"}\NormalTok{, }\StringTok{"16-node trees"}\NormalTok{,conf_matr}\OperatorTok{$}\NormalTok{overall[}\DecValTok{1}\NormalTok{],conf_matr_}\DecValTok{4}\OperatorTok{$}\NormalTok{overall[}\DecValTok{1}\NormalTok{],conf_matr_}\DecValTok{8}\OperatorTok{$}\NormalTok{overall[}\DecValTok{1}\NormalTok{],conf_matr_}\DecValTok{16}\OperatorTok{$}\NormalTok{overall[}\DecValTok{1}\NormalTok{]),}\DecValTok{4}\NormalTok{,}\DecValTok{2}\NormalTok{), }\DataTypeTok{col.names =} \KeywordTok{c}\NormalTok{(}\StringTok{"Deep"}\NormalTok{, }\StringTok{"Accuracy"}\NormalTok{))}
\end{Highlighting}
\end{Shaded}

\begin{verbatim}
## Error in matrix(c("Stump", "4-node trees", "8-node trees", "16-node trees", : object 'conf_matr_16' not found
\end{verbatim}

As we can see the difference between use stump or 16 node trees is not
to much. For this could be because with one stump we got a overfitting
to train dataset and is not need to increase the deep of nodes.

\end{document}
